%% $Id$
\documentclass{beamer}
\usepackage[T1]{fontenc}
\usepackage[isolatin]{inputenc}
\usepackage{beamerthemesplit}
\usepackage{fancyvrb}
\usepackage{verbatim}

\title{LDAP in der Linux Musterl�sung}
\author{R�diger Beck}
%\date{\today}

\begin{document}

\begin{frame}
  \titlepage
\end{frame}

%reset to nothing
\author{}
\date{}

\title{Postgresql und OpenLDAP}
\begin{frame}
  \begin{overprint}
  \begin{center}
    \textbf{\Large Gemeinsamkeiten}

    \vspace{4mm}

    \begin{tabular}{cp{5mm}c}
      \textbf{GNU/Linux System} && \textbf{Windows System}  \\[2mm]
      Software f�r Computer:    && Software f�r Computer:   \\
      Betriebsystem: \textcolor{red}{Linux}      
      && Betriebsystem: \textcolor{red}{Windows}   \\
      Textverarbeitung          && Textverarbeitung         \\
      Tabellenkalkulation       && Tabellenkalkulation      \\
      Webbrowser                && Webbrowser               \\
      Email-Programm            && Email-Programm           \\
      \ldots                    && \ldots                   \\
      80\,000 Personen-Jahre    && 80\,000 Personen-Jahre \\
    \end{tabular}
    % apple etwa dieselbe Ideologie wie Windows System
    % vermischung gewollt/nicht gewollt: firefox auf windows, ...
  \end{center}
  \end{overprint}
\end{frame}



\title{LDAP-Abfragen auf dem Server}
\begin{frame}
  \begin{overprint}
    \textbf{LDAP-Abfragen auf dem Server:} \par
        \begin{itemize}
        \item<2-> \texttt{ldapsearch -x} (Alles abfragen)
        \item<3-> Als admin:

         \texttt{ldapsearch -x -D cn=admin,dc=bszleo,dc=de -y /etc/ldap.secret} 

         \begin{itemize}
         \item<3-> \texttt{-D binddn} (Wer connectet?)
         \item<3-> \texttt{-y passwordfile} (Passwort?)
         \item<4-> Einschrankung auf einen User: \texttt{uid=name}
         \end{itemize}

        \item<5-> Alternativen: 
        \begin{itemize}
          \item<5-> \texttt{sophomorix-user -u user}
          \item<5-> \texttt{smbldap-usershow user}
          \item<5-> \texttt{id user}
          \item<5-> \texttt{getent passwd}, ...
        \end{itemize}

        \end{itemize}
  \end{overprint}
\end{frame}



\title{LDAP-Abfragen vom Schulnetz aus mit GUI}
\begin{frame}
  \begin{overprint}
    \textbf{LDAP-Abfragen vom Schulnetz aus mit GUI:} \par
        \begin{itemize}
        \item<2-> GUI-Beispiel: Luma (\texttt{aptitude install luma})
        \item<3-> Ldap Server Kofigurieren:
         \begin{itemize}
         \item<3-> Network: Hostname, Port, Encryption
         \item<3-> Authentification
         \begin{itemize}
            \item<4-> Anonymous Bind
            \item<5-> Simple,cn=admin,dc=bszleo,dc=de,Password
         \end{itemize}
         \item<6-> LDAP Options: Base DNs provided by Server
         \end{itemize}
        \end{itemize}
  \end{overprint}
\end{frame}

\title{Sinnvolle Plugins von \texttt{luma}}
\begin{frame}
  \begin{overprint}
    \textbf{Sinnvolle Plugins von \texttt{luma}:} \par
        \begin{itemize}
        \item<2-> Browser: Inhalt anzeigen lassen
        \item<3-> Search: Suchen im LDAP
        \item<4-> Schemabrowser: Muss-Inhalte,Kann-Inhalte
        \item<5-> Usermanagement: Nur zum ansehen n�tzlich
        \item<6-> Admin Utilities
        \item<7-> Verboten bei der paedML:
         \begin{itemize}
            \item<8-> Massive user creation
            \item<9-> 
         \end{itemize}
        \end{itemize}
  \end{overprint}
\end{frame}



\end{document}

%%% Local Variables: 
%%% mode: latex
%%% TeX-master: t
%%% End: 
